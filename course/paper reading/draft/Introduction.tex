There are two stable $Li$ isotopes naturally occurring, composed of lithium-6 and lithium-7, with the latter being much more abundant: about 92.5 percent of the atoms. Li has a wide use in the chenmical industry, such as the lithium ion battery, with highly reactive chemical property. Therefore, it is valuable to measure the physical properties of ${}^{7}Li$, e.g. magnetic moment.

As we all konow, mass is a fundamental property of fundamental particles, and the mass is made up of the mass of its constituents and the binding energy among them. It is important to measure the binding energy from a composite quantum system which contains various interaction, the strong, the electromagnetic and the weak force. The measurement can be regarded as a tests of Standard Model including the fundamental interactions and symmetries. Besides, if there existed difference of mass or magnetic moment between matter and antimatter, it would thus indicate something beyong Standard Model and might provide evidence for the matter/antimatter asymmetry in the universe. 

Initially, the Penning trap was first established as mass spectrometry by Hans Georg Dehmelt\cite{wiki.Penning} who awarded Nobel Prize in Physics about the developement of the ion trap technique in 1989. So far, Penning trap has had many fuctions including storing charged particles and cooling as well as its high-performace precision measuement with particularly well suited. Moreover, it has been used in the physical realization of quantum computation and quantum information processing by trapping qubits. Penning traps are used in many laboratories worldwide, including CERN, to store antimatter like antiprotons.  

In a Penning trap as Fig.2., in order to get the magnetic moment, there are some variables to measure, $\nu_{c}=q_{Li^{+}}B_{0}/(2\pi m_{Li^{+}})$ the cyclotron frequency, and $\nu_{L}=(g_/2)\nu_{c}$ is the spin-precession frequency, also called the Larmor frequency. Both frequencies are measured in the same magnetic field. The cyclotron frequency can be calculated by the Brown–Gabrielse invariance theorem, $\nu_c^{2}=\nu_{+}^2+\nu_{z}^2+\nu_{-}^2$, where $\nu_{+}$, $\nu_{z}$ and $\nu_{-}$ are characteristic oscillation frequencies of the trapped particle, the modified cyclotron frequency, the axial frequency and the magnetron frequency, respectively. We can obtain the magnetic moment through the equation:

\begin{figure}[]
\begin{center}
\includegraphics[width=0.7\textwidth]{plots/Introduction/penning-trap.png}
\end{center}
\caption{The double Penning trap overview, made up of presicion trap and analysis trap connected by five electrodes.}
\label{fig:penning-trap}
\end{figure}

\begin{align*}
\mu_{f}=g\frac{q}{2m}S \\
\mu_{N}=\frac{q_{p}\hbar}{2m_p}\\
\mu_{Li^{+}}=\frac{g}{2}\times \frac{m_{p}}{m_{Li^{+}}}=\frac{g}{2}\frac{m_p}{m_{Li^+}}
\end{align*}

where the equation $ mu_{f}=g\frac{q}{2m}S $ the calculation of spin magnetic moment of fermion in which S is the spin of particles, the $m_p$ is the mass of proton and $/mu_N$ the nuclear magneton. Therefore, get the value of $\frac{g}{2}=\frac{\nu_L}{\nu_c}$ and make use of the result of masses of proton and Li[2] can acquire the magentic moment of ${}^{7}Li^+$.  

In this work, firstly we make use of mass spectrometer consisting of three componets: an ion source, a mass analyzer, and a detector as the Fig.1 to obtain a $Li^{+}$ with specific isotope ${}^{7}Li^{+}$. After the ionization, the variety of ions can be identify by passing the magnetic field. The ions having the same ratio between charge and mass in a same track in the magnetic field can be extracted to get a pure ${}^{7}Li^{+}$ beam. Then inject the cloud-ionsinto the precision trap to obtain a single ion. About the equipment, we don't use a single trap but the double trap which can improve the result of measurement called analysis trap and precision trap as Fig.2. A strong magnetic field inhomogeneity is superimposed on the analysis trap, which is required to detect Li ion spin quantum transitions. However, The strong inhomogeneity broadens the width of the spin resonance, and ultimately limits the precision to the level of parts per million. An elegant solution to boost experimental precision is provided bu using precision trap (PT). It is in the centre of the PT we get a sigle ion from loaded ions using well established techniques. Furthermore, in the precision trap, where the magnetic field is homogeneous, which narrows the width of the Larmor resonance dramatically, and thus the precise frequency measurements are carried out.

\begin{figure}[]
\begin{center}
\includegraphics[width=0.7\textwidth]{plots/Introduction/mass-spectrometer.png}
\end{center}
\caption{The mass spectrometer setup.}
\label{fig:mass-spectrometer}
\end{figure}

In the Section II, we introduce the theory with more details. Section III presents the experiment setup, equipement introduction and operation. Section IV gives the conclusion.
